\documentclass[a4paper,american]{paper}
\usepackage[T1]{fontenc}
\usepackage[utf8]{inputenc}
\pagestyle{plain}
\usepackage{babel}
\usepackage{textcomp}
\usepackage{amsmath}
\usepackage{amsthm}
\usepackage{setspace}
\usepackage[unicode=true]{hyperref}
\usepackage{breakurl}
\usepackage{txfonts}
\usepackage{pxfonts}
\usepackage{tikz}
\usetikzlibrary{arrows.meta,positioning,calc}
\usepackage{tkz-graph}
\usepackage{graphicx}
\usepackage{float}
\usepackage{needspace}

\graphicspath{ {./images/} }

\makeatletter
\providecommand*{\code}[1]{\texttt{#1}}
\makeatother

\begin{document}

\title{An interactive demonstration of counterfactual truth conditions%
	\footnote{Further title proposals:
		(B) Creating an educational computer game about counterfactuals in terms of a centered system of spheres
		(C) Implementing a computer game illustrating the truth conditions of counterfactuals as variably strict conditionals
	}
}

\subtitle{Proposal for a Bachelor Thesis (v0.5, 3 August 2022)}

\author{%
	Andreas Paul Bruno Lönne\\
	\code{\href{mailto:loenne@campus.tu-berlin.de}{loenne@campus.tu-berlin.de}}
}

\institution{
	Technische Universität Berlin\\discourse
	Degree program: Bachelor Informatik / Computer Science
}

\maketitle

\section*{Problem: Representation of directed weighted graph-edges}
display numbers -> adds a lot of clutter
\\\\
create a hierarchy of different arrows, representing a handful of differently weighted edges -> cannot find good looking design using only one arrow property, when multiple properties are modulated the hierarchy becomes unclear
\\\\
scale thickness of arrows -> hard to gauge whose arrows value is higher in comparison, also ugly
\\\\
position small fuel/energy icon aside arrows -> adds some clutter, hard to position neatly
\\\\
only display numbers, when hovering over arrows -> users need to think forward to solve the levels and having to hover over arrows all the time is tedious
\\\\
Replace edges with linear paths of arrows, with the number of arrows equaling the weight of the edge -> creates hard layout problems, adds a lot of clutter
\\\\
Place big stars or planets along arrows, with their number equaling the edge weight (indicating stops on the journey) -> hard to distinguish stars from background stars, users have to tediously count stars, stars may be distracting clutter
\\\\
Add crossections to the arrow, like on a ruler -> doesnt look good, spacing of crosssections becomes inconsistent (feels weird)
\\\\
Add flowing animation from source to destination (speed, number of flyers) -> edgeweights should be discernible from a still image, speed is hard to gauge

\section*{Problem: UI-Design Learnings}
Everything a user needs needs to know has to be readily visible  \\\\
UI-Sprite Borders have to be slim to look nice \\\\
UI should be clear even in a still image

\section*{Problem: Learnings from Alpha-Testing}
Multiple Choice Fragebogen?

\section*{Problem: The Joys of Phaser}
Containers cannot be nested -> as such a lot of the ui-hierarchy had to be flattened and be built less modular (couldnt leverage relative positioning and property-inheritance as much) \\\\
Bitmapmasks produced weird darkened areas in areas with alpha of the masked sprite -> needed to convert all bitmapmasks to geometrymasks \\\\
rexUI a plugin developed for UI was time-consuming to use (searching docs all the time) and didnt offer enough flexibility to design original UI elements (Clipping mask of ScrollablePanel was forcably set to a rectangle ???) \\\\
Couldnt define level data in json format, since browser may not write files \\\\
tweens were a delight to use (are very powerful)

\section*{Definitions}
\subsection{Counterfactual formulas}
$Atoms = \{x,y,...\}$

$\Phi = \{\varphi, \psi,... \}$

$\varphi, \psi ::= \bot \mid x \mid \neg \varphi \mid \Box \varphi \mid \Diamond \varphi \mid \varphi \vee \psi \mid \varphi \wedge \psi \mid \varphi \boxright \psi$

\subsection{Worlds}
$W = \{w,v,...\}$

\subsection{Facts}
$F \colon W \rightarrow 2^{Atoms}$

\subsection{Similarity relation}
$\leadsto \colon W\times \mathbb{R} \times W$

\subsection{Accessible worlds}
$W_w = \{w'\mid w \overset{r}{\leadsto} w'\}$

\subsection{Truth conditions of counterfactual logic}
$w \vDash \bot$ is always false. \\
$w \vDash \top$ is always true. \\
$w \vDash x$ iff $x \in V(w)$. \\
$w \vDash \neg \varphi$ iff $w \nvDash \varphi$. \\
$w \vDash \varphi \vee \psi$ iff $(w \vDash \varphi$ or $w \vDash \psi)$ \\
$w \vDash \varphi \wedge \psi$ iff $(w \vDash \varphi$ and $w \vDash \psi)$ \\
$w \vDash \Box \varphi$ iff for every world $w'$, for which an $r$ with $w\overset{r}{\leadsto} w'$ exists, $w' \vDash \varphi$ holds true. \\
$w \vDash \Diamond \varphi$ iff a world $w'$ and an $r$ exist, such that $w\overset{r}{\leadsto} w'$ and $w' \vDash \varphi$ hold true. \\
$w \vDash \varphi \boxright \psi$, if no world $w'$ and $r$ exist, such that $w' \vDash \varphi$ and $w\overset{r}{\leadsto} w'$. \\
$w \vDash \varphi \boxright \psi$, if a world $w'$ and an $r$ exist, such that $w'\vDash \varphi$ and $w\overset{r}{\leadsto} w'$ and for each world $w*$, for which a $r*\leq r$ exists, such that $w\overset{r*}{\leadsto} w*$, $w*\vDash\psi\vee\neg\varphi$ holds true.
\subsection{Similarity graph}
\begin{equation}
	G = (V,E,F)\text{, such that }V \subseteq W\text{ and }E \subseteq \leadsto
\end{equation}

\section*{Rules of the semantic game}

\begin{figure}[H]
	\centering
	\begin{equation}
		(\bot ,w)_{a}\hspace{10pt}\text{Attacker wins}
	\end{equation}
	\begin{equation}
		(\top ,w)_{a}\hspace{10pt}\text{Defender wins}
	\end{equation}
	\begin{equation}
		(\neg\bot ,w)_{a} \rightarrow (\top ,w)_{a}
	\end{equation}
	\begin{equation}
		(\neg\top ,w)_{a} \rightarrow (\bot ,w)_{a}
	\end{equation}
	\begin{equation}
		(x,w)_{a}\xrightarrow{x\in F(w)}(\top ,w)_{d/a}
	\end{equation}
	\begin{equation}
		(x,w)_{a}\xrightarrow{x\not\in F(w)}(\bot ,w)_{d/a}
	\end{equation}
	\begin{equation}
		(\neg x,w)_{a}\xrightarrow{x\in F(w)}(\neg\top ,w)_{d/a}
	\end{equation}
	\begin{equation}
		(\neg x,w)_{a}\xrightarrow{x\not\in F(w)}(\neg\bot ,w)_{d/a}
	\end{equation}
	\begin{equation}
		(\neg\neg\varphi ,w)_{a}\rightarrow (\varphi ,w)_{d/a}
	\end{equation}
	\begin{equation}
		(\varphi\vee\psi ,w)_d\rightarrow (\varphi ,w)_{d/a}
	\end{equation}
	\begin{equation}
		(\varphi\vee\psi ,w)_d\rightarrow (\psi ,w)_{d/a}
	\end{equation}
	\begin{equation}
		(\neg (\varphi\vee\psi ),w)_a\rightarrow (\neg\varphi ,w)_{d/a}
	\end{equation}
	\begin{equation}
		(\neg (\varphi\vee\psi ),w)_a\rightarrow (\neg\psi ,w)_{d/a}
	\end{equation}
	
	\begin{equation}
		(\varphi\wedge\psi ,w)_a\rightarrow (\varphi ,w)_{d/a}
	\end{equation}
	\begin{equation}
		(\varphi\wedge\psi ,w)_a\rightarrow (\psi ,w)_{d/a}
	\end{equation}
	\begin{equation}
		(\neg (\varphi\wedge\psi ),w)_d\rightarrow (\neg\varphi ,w)_{d/a}
	\end{equation}
	\begin{equation}
		(\neg (\varphi\wedge\psi ),w)_d\rightarrow (\neg\psi ,w)_{d/a}
	\end{equation}	
	
	\begin{equation}
		(\varphi\boxright\psi ,w)_d\rightarrow (vac,\varphi ,w)_a
	\end{equation}
	\begin{equation}
		(vac,\varphi ,w)_a\xrightarrow{[w\overset{r}{\leadsto}w']}(\neg\varphi ,w')_{d/a}
	\end{equation}
	\begin{equation}
		(\varphi\boxright\psi ,w)_d\xrightarrow{[w\overset{r}{\leadsto}w']}(cf,\varphi ,\psi ,w,w',r)_a
	\end{equation}
	\begin{equation}
		(cf,\varphi ,\psi ,w,w',r)_a\rightarrow (\varphi ,w')_{d/a}
	\end{equation}
	\begin{equation}
		(cf,\varphi ,\psi ,w,w',r)_a\xrightarrow{[w\overset{r^*}{\leadsto}w^*,r^*\leq r]}(\neg\varphi\vee\psi ,w^*)_d
	\end{equation}
	\begin{equation}
		(\neg (\varphi\boxright\psi ),w)_a\rightarrow (vac,\varphi ,w)_d
	\end{equation}
	\begin{equation}
		(vac,\varphi ,w)_d\xrightarrow{[w\overset{r}{\leadsto}w']} (\varphi ,w')_{d/a}
	\end{equation}
	\begin{equation}
		(\neg (\varphi\boxright\psi ),w)_a\xrightarrow{[w\overset{r}{\leadsto}w']}(cf,\varphi ,\psi ,w,w',r)_d
	\end{equation}
	\begin{equation}
		(cf,\varphi ,\psi ,w,w',r)_d\rightarrow(\neg\varphi ,w')_{d/a}
	\end{equation}
	\begin{equation}
		(cf,\varphi ,\psi ,w,w',r)_d\xrightarrow{[w\overset{r^*}{\leadsto}w^*, r^*\leq r]}(\neg(\neg\varphi\vee\psi ),w^*)_a
	\end{equation}
	\caption{rules of the semantic game of counterfactuals}
	\label{fig:rules}
\end{figure}

\end{document}

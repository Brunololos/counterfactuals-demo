\documentclass[a4paper,american]{paper}
\usepackage[T1]{fontenc}
\usepackage[utf8]{inputenc}
\pagestyle{plain}
\usepackage{babel}
\usepackage{textcomp}
\usepackage[mathscr]{euscript}
\usepackage{amsmath}
\usepackage{amsthm}
\usepackage{setspace}
\usepackage[unicode=true]{hyperref}
\usepackage{breakurl}
\usepackage{txfonts}
\usepackage{pxfonts}
\usepackage{tikz}
\usetikzlibrary{arrows.meta,positioning,calc}
\usepackage{tkz-graph}
\usepackage{graphicx}
\usepackage{float}
\usepackage{needspace}

\graphicspath{ {./images/} }

\makeatletter
\providecommand*{\code}[1]{\texttt{#1}}
\makeatother

\begin{document}

\title{An interactive demonstration of counterfactual truth conditions%
	\footnote{Further title proposals:
		(B) Creating an educational computer game about counterfactuals in terms of a centered system of spheres
		(C) Implementing a computer game illustrating the truth conditions of counterfactuals as variably strict conditionals
	}
}

\subtitle{Proposal for a Bachelor Thesis (v0.5, 3 August 2022)}

\author{%
	Andreas Paul Bruno Lönne\\
	\code{\href{mailto:loenne@campus.tu-berlin.de}{loenne@campus.tu-berlin.de}}
}

\institution{
	Technische Universität Berlin\\discourse
	Degree program: Bachelor Informatik / Computer Science
}

\maketitle

\section*{Definitions}
\subsection{Counterfactual formulas}
$Atoms = \{x,y,...\}$

$\Phi = \{\varphi, \psi,... \}$

$\varphi, \psi ::= \bot \mid x \mid \neg \varphi \mid \Box \varphi \mid \Diamond \varphi \mid \varphi \vee \psi \mid \varphi \wedge \psi \mid \varphi \boxright \psi \mid \varphi \Diamondright \psi$

\subsection{Worlds}
$W = \{w,v,...\}$

\subsection{Facts}
$F \colon W \rightarrow 2^{Atoms}$

\subsection{Similarity relation}
$\leadsto \colon W\times \mathbb{R} \times W$

\subsection{Accessible worlds}
$W_w = \{w'\mid w \overset{r}{\leadsto} w'\}$

\newpage
\subsection{Truth conditions of counterfactual logic}
$w \vDash \bot$ is always false. \\
$w \vDash \top$ is always true. \\
$w \vDash x$ iff $x \in V(w)$. \\
$w \vDash \neg \varphi$ iff $w \nvDash \varphi$. \\
$w \vDash \varphi \vee \psi$ iff $(w \vDash \varphi$ or $w \vDash \psi)$ \\
$w \vDash \varphi \wedge \psi$ iff $(w \vDash \varphi$ and $w \vDash \psi)$ \\
$w \vDash \Box \varphi$ iff for every world $w'$, for which an $r$ with $w\overset{r}{\leadsto} w'$ exists, $w' \vDash \varphi$ holds true. \\
$w \vDash \Diamond \varphi$ iff a world $w'$ and an $r$ exist, such that $w\overset{r}{\leadsto} w'$ and $w' \vDash \varphi$ hold true. \\
$w \vDash \varphi \boxright \psi$, if no world $w'$ and $r$ exist, such that $w' \vDash \varphi$ and $w\overset{r}{\leadsto} w'$. \\
$w \vDash \varphi \boxright \psi$, if a world $w'$ and an $r$ exist, such that $w'\vDash \varphi$ and $w\overset{r}{\leadsto} w'$ and for each world $w*$, for which a $r*\leq r$ exists, such that $w\overset{r*}{\leadsto} w*$, $w*\vDash\psi\vee\neg\varphi$ holds true. \\
$w \vDash \varphi \Diamondright \psi$, iff a world $w'$ and an $r$ exist, such that $w\overset{r}{\leadsto} w'$ and $w' \vDash \varphi$ hold and for each world $w''$, for which an $r''$ exists, such that $w\overset{r''}{\leadsto}w''$ and $w'' \vDash \varphi$ hold true, a world $w*$ and an $r*$ exist, such that $r* \leq r''$ and $w\overset{r''}{\leadsto}w''$ and $w'' \vDash \varphi \wedge \psi$.
\subsection{Similarity graph}
\begin{equation}
	G = (V,E,F)\text{, such that }V \subseteq W\text{ and }E \subseteq \leadsto
\end{equation}

\section*{Rules of the semantic game}
\begin{figure}[H]
	\centering
	\begin{equation}
		(\top ,w)_{a}\hspace{10pt}\text{Attacker wins}
	\end{equation}
	\begin{equation}
		(\top ,w)_{d}\hspace{10pt}\text{Defender wins}
	\end{equation}
	\begin{equation}
		(\bot ,w)_{a}\hspace{10pt}\text{Attacker loses}
	\end{equation}
	\begin{equation}
		(\bot ,w)_{d}\hspace{10pt}\text{Defender loses}
	\end{equation}
	\caption{Win conditions}
	\label{fig:win_rules}
\end{figure}
	The win conditions for attacker and defender are identical.
	A player who reaches a top-symbol wins and a player who reaches a bottom symbol loses.
	Since attacker and defender are treated equally in this game formulation i will introduce the shorthands $e\in\{ a, d\}$ and $o = \{ a, d\}\setminus{}e$ to avoid the duplication of every rule.
\begin{figure}[H]
	\centering
	\begin{equation}
		(x,w)_{e}\xrightarrow{x\in F(w)}(\top ,w)_{e}
	\end{equation}
	\begin{equation}
		(x,w)_{e}\xrightarrow{x\not\in F(w)}(\bot ,w)_{e}
	\end{equation}
	\begin{equation}
		(\neg\varphi ,w)_{e}\rightarrow (\varphi ,w)_{o}
	\end{equation}
	\caption{Atom resolution \& Negation}
	\label{fig:neg_rules}
\end{figure}
	The negation is resolved by switching the active player.
\begin{figure}[H]
	\centering
	\begin{equation}
		(\varphi\vee\psi ,w)_e\rightarrow (\varphi ,w)_{e}
	\end{equation}
	\begin{equation}
		(\varphi\vee\psi ,w)_e\rightarrow (\psi ,w)_{e}
	\end{equation}
	\begin{equation}
		(\varphi\wedge\psi ,w)_e\rightarrow (And, \varphi\wedge\psi ,w)_{o}
	\end{equation}
	\begin{equation}
		(And, \varphi\wedge\psi ,w)_e\rightarrow (\varphi ,w)_{o}
	\end{equation}
	\begin{equation}
		(And, \varphi\wedge\psi ,w)_e\rightarrow (\psi ,w)_{o}
	\end{equation}
	\caption{Disjunction \& Conjunction}
	\label{fig:neg_rules}
\end{figure}
	According to the disjunctions truth conditions the active-("proving")-player may choose which subformula to evaluate further. Conversely the conjunctions truth conditions are modelled by allowing the nonactive-("disproving")-player to make that choice instead. \\\\
	
\begin{figure}[H]
	\centering
	\begin{equation}
		(\Diamond\varphi ,w)_e\xrightarrow{[w\overset{r}{\leadsto}w']} (\varphi ,w')_{e}
	\end{equation}
	\begin{equation}
		(\Diamond\varphi ,w)_e\rightarrow (\bot ,w)_{e}\text{,}
	\end{equation}
	if no world $w'$ and $r$ exist, such that $w\overset{r}{\leadsto}w'$.
	\begin{equation}
		(\Box\varphi ,w)_e\rightarrow (Nec,\Box\varphi ,w')_{o}
	\end{equation}
	\begin{equation}
		(\Box\varphi ,w)_e\rightarrow (\top ,w)_{e}
	\end{equation}
	if no world $w'$ and $r$ exist, such that $w\overset{r}{\leadsto}w'$.
	\begin{equation}
		(Nec, \Box\varphi ,w)_e\xrightarrow{[w\overset{r}{\leadsto}w']} (\varphi ,w')_{o}
	\end{equation}
	\caption{Possibility \& Necessity}
	\label{fig:modal_operator_rules}
\end{figure}

	Regarding the case that no world is accessible the modal possibility operator evaluates to false, since it stipulates the existence of a world. The necessity operator on the other hand only stipulates its subformula to hold at each accessible world and would thus be vacuously true.
\begin{figure}[H]
	\centering
	\begin{equation}
		(\varphi\Diamondright\psi ,w)_e\xrightarrow{[w\overset{r}{\leadsto}w']} (Cf,\varphi\Diamondright\psi ,w,w',r)_o
	\end{equation}
	\begin{equation}
		(\varphi\Diamondright\psi ,w)_e\rightarrow (\bot ,w)_e
	\end{equation}
	if no world $w'$ and $r$ exist, such that $w\overset{r}{\leadsto}w'$.
	\begin{equation}
		(Cf,\varphi\Diamondright\psi ,w,w',r)_e\xrightarrow{[w\overset{r^*}{\leadsto}w^*,r^*< r]} (\varphi ,w*)_o
	\end{equation}
	\begin{equation}
		(Cf,\varphi\Diamondright\psi ,w,w',r)_e\rightarrow (\varphi\wedge\psi ,w')_o
	\end{equation}
	\caption{Counterfactual might}
	\label{fig:counterfactual_might_rules}
\end{figure}
	The truth conditions of the counterfactual might operator are as described in figure~\ref{fig:counterfactual_might}.
	
	I have reduced Lewis' truth conditions to the existence of a $\varphi$ \& $\psi$-world among the closest $\varphi$-worlds. This is derived from Lewis' truth conditions as follows. \\
	Suppose no $\varphi$-world is accessible from $w$, that is, no $r$ and $w'$ exist such that $w\overset{r}{\leadsto}w'$ and $w'\vDash\varphi$. Then it follows that no $r$ and $w'$ exist such that $w\overset{r}{\leadsto}w'$ and $w'\vDash\varphi\wedge\psi$. \\
	Now suppose that a closest $\varphi$-world $w'$ to $w$ exists, that is, some $r$ and $w'$ exist such that $w\overset{r}{\leadsto}w'$ and $w'\vDash\varphi$ and no $r*$ and $w*$ exist such that $w\overset{r*}{\leadsto}w*$, $w*\vDash\varphi$ and $r*<r$. \\
	Then if a $\varphi$ \& $\psi$-world $w''$ exists such that $w\overset{r}{\leadsto}w''$, $w''\vDash\varphi\wedge\psi$, $w''$ is included in every sphere $w'$ is included in. Since we supposed that $w'$ is the closest $\varphi$-world to $w$, we can conclude by the nesting property of centered systems of spheres, that the set of spheres centered on $w$, that contain $w'$, is the same set as the set of spheres centered on $w$, that contain at least one $\varphi$-world. Thus every sphere centered on $w$, that contains a $\varphi$-world also contains a $\varphi\wedge\psi$-world. \\
	On the other hand if no $\varphi$ \& $\psi$-world $w''$ exists such that $w\overset{r}{\leadsto}w''$, $w''\vDash\varphi\wedge\psi$, then there simply exists the sphere delimited by $w'$ and centered on $w$, that contains at least a $phi$-world we named $w'$ and no $\varphi\wedge\psi$-world. \\
	
	Here just a few quick comments about the Rules: \\
	Rule (20) makes the active-("proving")-player lose in case no worlds are accessible at all. \\
	Rule (19) is the active players choice of a sphere of accessibility, by choosing a deliminating world. The deliminating world has to be a $\varphi\wedge\psi$-world and a closest $\varphi$-world to win the game.\\
	Rule (21) gives the previous non-active-player the opportunity to disprove the chosen world is a closest $\varphi$-world. \\
	Rule (22) lets the previous non-active-player evaluate the chosen world on whether it is a $\varphi\wedge\psi$-world.
\begin{figure}[H]
	\centering
	\begin{equation}
		(\varphi\boxright\psi ,w)_e\rightarrow (Would,\varphi\boxright\psi ,w)_o
	\end{equation}
	\begin{equation}
		(Would,\varphi\boxright\psi ,w)_e\rightarrow (\bot ,w)_e
	\end{equation}
	if no world $w'$ and $r$ exist, such that $w\overset{r}{\leadsto}w'$.
	\begin{equation}
		(Would,\varphi\boxright\psi ,w)_e\xrightarrow{[w\overset{r}{\leadsto}w']} (Cf,\varphi\boxright\psi ,w,w',r)_o
	\end{equation}
	\begin{equation}
		(Cf,\varphi\boxright\psi ,w,w',r)_e\xrightarrow{[w\overset{r^*}{\leadsto}w^*,r^*< r]} (\varphi ,w*)_e
	\end{equation}
	\begin{equation}
		(Cf,\varphi\boxright\psi ,w,w',r)_e\rightarrow (\neg\varphi\vee\psi ,w')_e
	\end{equation}
	\caption{Counterfactual would}
	\label{fig:counterfactual_would_rules}
\end{figure}
	The rules for the counterfactual would are defined analogously to the counterfactual might rules and stated simplest by the question "Is no $\varphi$ \& $\neg\psi$-world among the closest $\varphi$-worlds?". \\
	The non-active-("disproving")-player becomes active and has to choose a $\varphi$ \& $\neg\psi$-world, thats also a closest $\varphi$-world to win. Then the formerly active-("proving")-player becomes active once more and may choose to either contend that the chosen world is a closest $\varphi$-world (Rule (26)) or claim that the chosen world is not a $\varphi$ \& $\neg\psi$-world. \\

	Note that the game is defined in a way to have the defender i.e. player always start as the active player and prove a formula. Should the defender not start as the active player their goal simply changes to disprove the formula instead of proving it.
	\newpage
\begin{figure}[H]
	\centering
	$\phi\Diamondright\psi$ is true at a world i (according to a system of spheres \$) if and only if both
	\begin{itemize}
	\item[(1)] some $\phi$-world belongs to some sphere $S$ in $\$_i$, and
	\item[(2)] every sphere $S$ in $\$_i$ that contains at least one $\phi$-world contains at least one world where $\phi$ \& $\psi$ holds.
	\end{itemize}
	\caption{Lewis' counterfactual might truth conditions}
	\label{fig:counterfactual_might}
\end{figure}
\begin{figure}[H]
	\centering
	$\phi\boxright\psi$ is true at a world i (according to a system of spheres \$) if and only if either
	\begin{itemize}
	\item[(1)] no $\phi$-world belongs to any sphere $S$ in $\$_i$, or
	\item[(2)] some sphere $S$ in $\$_i$ does contain at least one $\phi$-world, and $\phi\supset\psi$ holds at every world in $S$.
	\end{itemize}
	\caption{Lewis' counterfactual would truth conditions}
	\label{fig:counterfactual_would}
\end{figure}
\begin{figure}[H]
	\centering
	Let $\$$ be an assignment to each possible world $i$ of a set $\$_i$ of sets of possible worlds. Then $\$$ is called a (centered) system of spheres, and the members of each $\$_i$ are called spheres around $i$, if and only if, for each world $i$, the following conditions hold.
	\begin{itemize}
	\item[(C)] $\$_i$ is centered on $i$; that is, the set $\{i\}$ having $i$ as its only member belongs to $\$_i$.
	\item[(1)] $\$_i$ is nested; that is, whenever $S$ and $T$ belong to $\$_i$, either $S$ is included in $T$ or $T$ is included in $S$.
	\item[(2)] $\$_i$ is closed under unions; that is, whenever $\mathscr{S}$ is a subset of $\$_i$ and $\bigcup\mathscr{S}$ is the set of all worlds $j$ such that $j$ belongs to some member of $\mathscr{S}$, $\bigcup\mathscr{S}$ belongs to $\$_i$.
	\item[(3)] $\$_i$ is closed under (nonempty) intersections; that is, whenever $\mathscr{S}$ is a nonempty subset of $\$_i$ and $\bigcap\mathscr{S}$ is the set of all worlds $j$ such that $j$ belongs to every member of $\mathscr{S}$, $\bigcap\mathscr{S}$ belongs to $\$_i$.
	\end{itemize}
	\caption{Lewis' (centered) system of spheres}
	\label{fig:system_of_spheres}
\end{figure}
\end{document}

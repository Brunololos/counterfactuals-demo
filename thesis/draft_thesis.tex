\documentclass[a4paper,american]{paper}
\usepackage[T1]{fontenc}
\usepackage[utf8]{inputenc}
\pagestyle{plain}
\usepackage{babel}
\usepackage{textcomp}
\usepackage[mathscr]{euscript}
\usepackage{amsmath}
\usepackage{amsthm}
\usepackage{setspace}
\usepackage[unicode=true]{hyperref}
\usepackage{breakurl}
\usepackage{txfonts}
\usepackage{pxfonts}
\usepackage{tikz}
\usetikzlibrary{arrows.meta,positioning,calc}
\usepackage{tkz-graph}
\usepackage{graphicx}
\usepackage{float}
\usepackage{needspace}

\graphicspath{ {./images/} }

\makeatletter
\providecommand*{\code}[1]{\texttt{#1}}
\makeatother

\theoremstyle{definition}\newtheorem{definition}{Definition}
\hyphenation{counterfactual}

\begin{document}

\title{An interactive demonstration of counterfactual truth conditions}

\subtitle{Bachelor Thesis}

\author{%
	Andreas Paul Bruno Lönne\\
	\code{\href{mailto:loenne@campus.tu-berlin.de}{loenne@campus.tu-berlin.de}}
}

\institution{
	Technische Universität Berlin\\discourse
	Degree program: Bachelor Informatik / Computer Science
}

\maketitle

\section{The semantic game of counterfactuals}
\subsection{Counterfactual logic}
Counterfactual logic is an extension of propositional modal logic, that adds the binary operators $\boxright$, $\Diamondright$ and atomic formulas. The introduction of the {\it counterfactual~would}~$\boxright$ and {\it counterfactual~might}~$\Diamondright$ operators \cite{lewis_counterfactuals_1973} aims to model the counterfactual conditional constructions of ordinary language, while atomic formulas serve as a way to represent states of affairs that may or may not be the case at a world in possible world semantics.
\begin{definition}[\textbf{Counterfactual formula}]
\label{pythagorean}
May $Atoms = \{x,y,...\}$ be the set of atomic formulas and $\Phi = \{\varphi, \psi,... \}$ the set of all formulas. Formulas are of the form: \\
$\varphi, \psi ::= \bot \mid x \mid \neg \varphi \mid \Box \varphi \mid \Diamond \varphi \mid \varphi \vee \psi \mid \varphi \wedge \psi \mid \varphi \boxright \psi \mid \varphi \Diamondright \psi$.
\end{definition}
\begin{definition}[Atoms]
$Atoms = \{ x, y, ...\}$ is an infinite set of symbols.
\end{definition}
\begin{definition}[Alphabet of counterfactual logic]
The alphabet of counterfactual logic $A = \{\bot ,\top ,\neg ,\vee ,\wedge ,\Diamond ,\Box ,\Diamondright ,\boxright\}\cup Atoms$ is an infinite set of symbols.
\end{definition}
\begin{definition}[counterfactual formula]
A counterfactual formula is a finite sequence of symbols from the alphabet $A$.
\end{definition}
\begin{definition}[well-formedness]
A counterfactual formula is called well-formed, iff it takes the form described by $\varphi, \psi ::= \bot \mid x \mid \neg \varphi \mid \Box \varphi \mid \Diamond \varphi \mid \varphi \vee \psi \mid \varphi \wedge \psi \mid \varphi \boxright \psi \mid \varphi \Diamondright \psi$
\end{definition}
Subsequently every counterfactual formula is assumed to be well-formed.
\begin{definition}[Set of all formulas]
Let $\Phi$ be the set of all well-formed counterfactual formulas.
\end{definition}

\bibliographystyle{alphaurl}
\nocite{*}
\bibliography{thesis}

\end{document}
